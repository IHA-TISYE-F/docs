%Interaction between two individual elements in a system are %affected by the interface which is connecting the two parts.
There are different ways in which a system interact with it's environment and the other systems. The interaction happening at the various boundaries are called the system's external interfaces. The boundaries between individual components inside the system are called system's internal interfaces.\\
The external and internal identification can fall into different types such as: electrical, mechanical, real-time data transfer and storage-and-retrieval of data. All the interfaces illustrated in figure. ~\ref{fig:sigFlowDiagram}.
\begin{figure}[h]
	\centering
	\includegraphics[scale=0.5]{./images/SignalFlowDiagramDDD}\\
	\caption{Signal Flow Diagram}
    \label{fig:sigFlowDiagram}
\end{figure}

\subsubsection{External interfaces}
External interfaces are the interaction between the missile and the sensors, chaffs/flares and the dispensers, cockpit control unit and the aircraft computer, and lastly the pilot an the computer.
\begin{itemize}
\item {Interface identification and diagrams.}\\
The system interfaces listed and identified on table. ~\ref{tab:External}.


\begin{center}
\begin{tabular}{ | p{2cm} | l | p{2.3cm} | p{2.3cm} | l | p{1cm} |}
\hline
 \textbf{Interface name} & \textbf{Identification} & \textbf{Endpoint A} & \textbf{Endpoint B} & \textbf{Standard}\\ \hline
Computer Pilot interaction & E-IF-COMPPIL & Mission computer/ Pilot & Pilot/Mission computer & \\ \hline
 Computer control & E-IF-COMPCTRL & Cockpit unit/ Mission computer &Mission computer/Cockpit unit & \\ \hline
Missile sensor input & E-IF-MISSENS & Cockpit unit & MWS & \\ \hline
Chaffs/flares missile disturbance& E-IF-CHAFMISS & Chaffs and flares & Incoming missile & \\ \hline
Dispenser Chaffs control & E-IF-DISCHAF & Dispenser & Chaffs and flares & \\ \hline
\end{tabular}
\end{center}

\item {Project-unique identifier of interface}\\
\begin{sidewaystable}
\begin{tabular}{ l l l l l l l }
\hline
%Environmental&&&&&&&&\\
&Type&Interaction medium&data element&communication methods&protocols&physical compatibility\\
\hline
Incoming missile&&&&&&\\
\hline
Chaffs and flares&&&&&&\\
\hline
Aircraft Mission Computers&&&&&&\\
\hline
System Operators&&&&&&\\
\hline
Maintenance&&&&&&\\
\hline
Support&&&&&&\\
\hline
System Housing&&&&&&\\
\hline
Shipping and handling&&&&&&\\
\hline
\end{tabular}
\caption{External Interface Elements}
\end{sidewaystable}
%Electrical&Mechanical&Hydraulic
\end{itemize}

<<<<<<< HEAD

=======
>>>>>>> dd97819ecd1e7e1a41804c58d23739ef828ceca4
\subsubsection{Internal interfaces}
This section describes the internal interfaces. The system interfaces can be seen on figure \ref{fig:sigOverviewIdentification}. The internal interfaces are:

\begin{itemize}
<<<<<<< HEAD
\item{Interface identification and diagrams.}\\
blaa
\item{Project-unique identifier of interface}\\

\end{itemize}
=======
\item I-IF-MWSCTRL
\item I-IF-DISCTRL
\item I-IF-PODPWR
\end{itemize}

\paragraph{I-IF-MWSCTRL}

\begin{center}
\begin{tabular}{ | p{2cm} | l | p{2.3cm} | p{2.3cm} | l | p{1cm} |}
\hline
 \textbf{Interface name} & \textbf{Identification} & \textbf{Endpoint A} & \textbf{Endpoint B} & \textbf{Standard}\\ \hline

 MWS Control & I-IF-MWSCTRL & Cockpit unit & MWS & MIL-STD-1553-B\\ \hline

\end{tabular}
\end{center}

Some description...
\\
The data 
\begin{itemize}
\item Threat data (MWS $\rightarrow$ Cockpit unit)
	\begin{itemize}
	\item Direction relative to north
	\item Size
	\item Velocity
	\end{itemize}
\item Aircraft navigation data (Cockpit unit $\rightarrow$ MWS)
	\begin{itemize}
	\item Altitude
	\item Heading
	\item Position data
	\end{itemize}
\end{itemize}
The physical layer is defined by the MIL-STD-1553-B standard.


\paragraph{I-IF-DISCTRL}

\begin{center}
\begin{tabular}{ | p{2cm} | l | p{2.3cm} | p{2.3cm} | l | p{1cm} |}
\hline
 \textbf{Interface name} & \textbf{Identification} & \textbf{Endpoint A} & \textbf{Endpoint B} & \textbf{Standard}\\ \hline

 Dispenser Control & IF-DISCTRL & Cockpit unit & Dispenser assembly & MIL-STD-1553-B\\ \hline
 
\end{tabular}
\end{center}

Some description...\\
Something about data:
\begin{itemize}
\item direction to fire
\item what to fire (chaffs/flares)
\item pattern to fire
\item fire command
\end{itemize}

The physical layer is defined by the MIL-STD-1553-B standard.

\paragraph{I-IF-PODPWR}

\begin{center}
\begin{tabular}{ | p{2cm} | l | p{2.3cm} | p{2.3cm} | l | p{1cm} |}
\hline
 \textbf{Interface name} & \textbf{Identification} & \textbf{Endpoint A} & \textbf{Endpoint B} & \textbf{Standard}\\ \hline

 Pod power control & IF-PODPWR & Cockpit unit & Dispenser assembly and MWS & N/A\\ \hline
 
\end{tabular}
\end{center}

This analog signal connects from the cockpit unit to the dispenser assembly and the MWS in the pod. When asserted, this signal enables power to the dispenser assembly and MWS. When not asserted, the power is off.
>>>>>>> dd97819ecd1e7e1a41804c58d23739ef828ceca4
