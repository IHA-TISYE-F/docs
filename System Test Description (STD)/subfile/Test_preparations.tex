\noindent
This section contains the preparations required for all tests.

\paragraph{ReqNo1-T}\mbox{}\\ % Lasse
Requirement: \textit{The system shall comprise at least three modes, manual, semi-automatic and automatic.}
\\
The ability for the pilot to select each mode is tested.

	\subparagraph{Preparations needed:}
	\begin{itemize}
	\item None.
	\end{itemize}


\paragraph{ReqNo2-T}\mbox{}\\ % Lars N
Requirement: \textit{Manual mode shall dispense the program selected by the pilot. The pilot may select payload, and dispense direction as defined by UR-2.}
\\
The operator will dispense the desired payload in the desired direction. All combinations of payload and direction is tested.
\\
	\subparagraph{Preparations needed:}
	\begin{itemize}
	\item Manual mode is selected.
	\end{itemize}


\paragraph{ReqNo3-T}\mbox{}\\ % Sergiu
Requirement: \textit{Semi automatic shall initiate an intelligent threat response upon consent from the pilot.}
\\
This test verifies that the cockpit unit, in semi automatic mode, can send a signal that requests the pilots consent before deploying the payload. All combinations of payload and direction is tested
\\
	\subparagraph{Preparations needed:}
	\begin{itemize} 
	\item Semi automatic mode is selected
	\item The pod is loaded with chaffs or flares
	\item The plane is airborne
	\end{itemize}


\paragraph{ReqNo4-T}\mbox{}\\ % Lars J
Requirement: \textit{Automatic mode shall initiate an intelligent threat re-
sponse without pilot interaction.}\\
This test verifies that the cockpit unit, in automatic mode, can send a signal to the dispenser.

	\subparagraph{Preparations needed:}
	\begin{itemize}
	\item Automatic mode is selected.
	\end{itemize}
	
	\subparagraph{Equipment required:}
	\begin{itemize}
	\item The cockpit unit and the dispenser must be attached to the aircraft and the dispenser must be loaded with chaffs and flares.
	\end{itemize}	

\paragraph{ReqNo5-T}\mbox{}\\ % Ivan
Requirement: \textit{The pod shall include a minimum of eight standard magazines.}\\
This test verifies that the pod includes a minimum of eight standard magazines.
	\subparagraph{Equipment required:}
	\begin{itemize}
	\item Pod.
	\item Eight standard magazines.
	\end{itemize}
	
	\subparagraph{Preparations needed:}
	\begin{itemize}
	\item None.
	\end{itemize}

\paragraph{ReqNo6-T}\mbox{}\\ % Fatemeh
Requirement: \textit{The pod shall be able to dispense forwards, downwards and sideways.}\\
This test verifies the mechanical capability and reliability. The test applies to the pod components, meaning magazines and dispensers.
	\subparagraph{Preparations needed:}
	\begin{itemize}
	\item Accurate, high precision motor control needed for all ways movements.
	\item High precision time calculator needs for measuring the delay and accuracy of the synchronization between the components.
	\end{itemize}

\paragraph{ReqNo7-T}\mbox{}\\ % Lasse
Requirement: \textit{The cockpit unit shall be able to power ON and OFF the dispensing system and the MWS.}
\\
The MWS and dispenser assembly is turned on and off. In each state the the result is verified.

	\subparagraph{Preparations needed:}
	\begin{itemize}
	\item The system main power source must be turned on.
	\end{itemize}


\paragraph{ReqNo8-T}\mbox{}\\ % Lars N
Requirement: \textit{The system shall be able to dispense a minimum of two payloads within 0.1 sec.}
\\
Using manual mode, the operator tries to dispense two payloads simultaneously. It is measured whether the payloads are dispensed within 0.1 sec.
\\
	\subparagraph{Preparations needed:}
	\begin{itemize}
	\item A stop-watch must be provided.
	\end{itemize}

\paragraph{ReqNo9-T}\mbox{}\\ % Sergiu
Requirement: \textit{The system shall be able to dispense a pattern of payloads programmable by the customer.}
\\
This test verifies that the program outputs in the desired pattern that was programmed by the customer. Afterwards it is loaded and tested on the pod to see the it dispense in the desired pattern.
\\
	\subparagraph{Preparations needed:}
	\begin{itemize}
	\item A voltmeter that tests the voltage that runs through the pod.
	\item The pod is loaded with chaffs or flares
	\item The plane is airborne
	\end{itemize}

\paragraph{ReqNo11-T}\mbox{}\\ % Lars J
Requirement: \textit{The System shall be able to erase prior defence patterns and usage statistics upon receiving the string 'zeroize' from the mission computer.}\\
This test verifies that the system is able to erase prior defence patterns and usage statistics.
	\subparagraph{Preparations needed:}
	\begin{itemize}
	\item The cockpit unit must be connected to the mission computer.
	\end{itemize}

	\subparagraph{Equipment required:}
	\begin{itemize}
	\item None.
	\end{itemize}

\paragraph{ReqNo12-T}\mbox{}\\ % Ivan
Requirement: \textit{The cockpit unit shall communicate with the MWS via a MIL-STD-1553-B data bus.}\\
This test verifies that the cockpit unit and the MWS communicate with the MIL-STD-1553-B data bus.
	\subparagraph{Preparations needed:}
	\begin{itemize}
	\item None.
	\end{itemize}

\paragraph{ReqNo13-T}\mbox{}\\ % Fatemeh
Requirement: \textit{Threats shall be transmitted to the aircraft mission computer in body frame format (relative to aircraft) for displaying purposes.}\\
The test includes simulating different types of threats from different directions and testing the reliability of the aircraft mission computer for displaying the threat.
	\subparagraph{Preparations needed:}
	\begin{itemize}
	\item Threat simulation should be comparable to the real case threat
	\item Different directions relative to north and different velocities should be provided for the simulated threat.
	\item Different displaying tools, such as audio and visual needed
	\item Time calculator needs for calculating the time takes for transmitting the threat from MSW to display.
	\item Pilot reaction time for testing different display methods
	\end{itemize}

\paragraph{ReqNo14-T}\mbox{}\\ % Lasse
Requirement: \textit{Threat information will be provided by the Electronics Control Unit (ECU).}
\\
It is tested that threat information is supplied by the ECU when a threat is simulated.

	\subparagraph{Preparations needed:}
	\begin{itemize}
	\item A method for simulating threats must be supplied.
	\item A method for reading the interface I-IF-MWSCTRL defined in section 6 of F-DDD-2014-V1.
	\end{itemize} 


\paragraph{ReqNo15-T}\mbox{}\\ % Lars N
Requirement: \textit{The system shall provide the aircraft mission computer
with status information and built-in test results.}
\\
Test-software installed on the mission computer requests status information and built-in test results from the system. The test-software verifies that the received data is correct.
\\
	\subparagraph{Preparations needed:}
	\begin{itemize}
	\item Test-software for the mission computer must be developed.
	\end{itemize} 

\paragraph{ReqNo16-T}\mbox{}\\ % Sergio=u
Requirement: \textit{The system shall interface the aircraft intercom system to provide audio cues and warnings.}
\\
This test verifies that the aircraft intercom system receives information from the system.
\\
	\subparagraph{Preparations needed:}
	\begin{itemize}
	\item None.
	\end{itemize} 


\paragraph{ReqNo17-T}\mbox{}\\ % Lars J
Requirement: \textit{The system status on individual LRU level shall be
provided by cockpit unit.}\\
This test verifies that the Cockpit Unit can provide status updates from the Pod.\\

	\subparagraph{Preparations needed:}
	\begin{itemize}
	\item The Pod, dispenser and magazines must be attached to the plane.
	\end{itemize}

	\subparagraph{Equipment required:}
	\begin{itemize}
	\item None.
	\end{itemize}

\paragraph{ReqNo18-T}\mbox{}\\ % Ivan
Requirement: \textit{The MWS must receive navigation data from the aircraft mission computer with a latency of no more than
10 ms. Navigation data includes aircraft attitude,heading, altitude and GPS data.}\\

This test verifies that the MWS receives navigation data from the aircraft mission computer within the allowed time range.

	\subparagraph{Preparations needed:}
	\begin{itemize}
	\item Test navigation data including attitude, heading, altitude and GPS data must be loaded on the aircraft mission computer.
	\item A high precision stop-watch must be provided.
	\end{itemize}

\paragraph{ReqNo20-T}\mbox{}\\ % Fatemeh
Requirement: \textit{The cockpit unit shall communicate with the mission computer via a MIX-STD-1553-B data bus.}\\
Test includes sending commands to the mission computer and making it understandable for the pilot.
	\subparagraph{Preparations needed:}
	\begin{itemize}
	\item The communication of cockpit unit with Pod should have been done and tested before hand.
	\item The mission computer need compatibility for data bus.
	\end{itemize}

\paragraph{ReqNo21-T}\mbox{}\\ % Lasse
Requirement: \textit{Introduction of the system may not compromise the operation of the current weapon systems.}
\\
Current weapon systems are tested with the self-protection suite installed.

	\subparagraph{Preparations needed:}
	\begin{itemize}
	\item All current weapon systems must be available for testing.
	\end{itemize} 
	

\paragraph{ReqNo22-T}\mbox{}\\ % Lars N
Requirement: \textit{The system shall include a hardware implemented safety interlock to prevent dispensing on ground.}
\\
It is assured that the aircraft is touching the ground. Then it is tried to dispense payloads. It is observed whether the system dispenses the payload or not.
\\
	\subparagraph{Preparations needed:}
	\begin{itemize}
	\item None.
	\end{itemize} 

\paragraph{ReqNo23-T}\mbox{}\\ % Sergiu
Requirement: \textit{The hardware implemented safety lock shall be activated when the landing gear is on the ground.}
\\
This test verifies that the safety lock activates and remains active while the
landing gear is on the ground.
\\
	\subparagraph{Preparations needed:}
	\begin{itemize}
	\item None.
	\item The landing gear is on the ground.
	\item The pod is empty, for safety reasons.
	\end{itemize} 
%Note to Sergiu:
%The hardware interlock engages when the landing gear is on the ground and not at a given altitude. This is of course also the case for the tex file Test_descriptions.tex

\paragraph{ReqNo24-T}\mbox{}\\ % Lars J
Requirement: \textit{The system shall be able to erase sensitive data upon input from a discrete zeroize signal from the aircraft.}\\

This test verifies that the system can erase sensitive data.
	\subparagraph{Preparations needed:}
	\begin{itemize}
	\item The cockpit unit main power must be turned on.
	\end{itemize}

	\subparagraph{Equipment required:}
	\begin{itemize}
	\item The cockpit unit and mission computer must be connected.
	\end{itemize}

\paragraph{ReqNo25-T}\mbox{}\\ % Ivan
Requirement: \textit{The zeroize signal shall be received by the cockpit unit.}\\

This test verifies that the zeroize signal is received by the cockpit unit.
	\subparagraph{Preparations needed:}
	\begin{itemize}
	\item None.
	\end{itemize}

\paragraph{ReqNo26-T}\mbox{}\\ % Fatemeh
Requirement: \textit{The magasines shall be stored at no lower than -10 degrees Celcius and no higher than 70 degrees Celcius.}
The temperature at which the magasines are stored shall be verified to be between -10 degrees Celcius and 70 degrees Celcius.

	\subparagraph{Preparations needed:}
	\begin{itemize}
	\item None.
	\end{itemize}

\paragraph{ReqNo27-T}\mbox{}\\ % Lasse
Requirement: \textit{The pod structure must be functional when exposed to steady state acceleration levels of 4g forward, 2.5g backward, 22g upward or 10g downward.}
\\
The pod structure is subjected to steady accelerations, and then inspected for damages that may reduce functionality.

\subparagraph{Preparations needed:}
	\begin{itemize}
	\item A test setup to create the required steady state accelerations must be provided.
	\end{itemize} 
	

\paragraph{ReqNo28-T}\mbox{}\\ % Lars N
Requirement: \textit{The total weight of pod cannot exceed 270 kg.}
\\
The total weight of pod measured is measured using a weighing scale. It is noted whether the weight is above or below 270 kg.
\\
	\subparagraph{Preparations needed:}
	\begin{itemize}
	\item Weighing scale must be provided.
	\end{itemize} 

\paragraph{ReqNo29-T}\mbox{}\\ % Sergiu
Requirement: \textit{The pod shall be operational at temperatures of maximum 134 degree celsius on outer skin and 152 degree celsius on leading edge for maximum 3 minutes.}
\\
This test verifies that the pod is operational at temperatures of up to 134 celsius on the outer skin. It also checks that the pod is operational after it was subjugated to a temperature of 152 degrees for 3 minutes.
\\
	\subparagraph{Preparations needed:}
	\begin{itemize}
	\item Means of recording time and two temperature sensors, one placed on the outer skin and one placed on the leading edge.
	\item The aircraft is airborne.
	\end{itemize} 


\paragraph{ReqNo41-T}\mbox{}\\ % Lars J
Requirement: \textit{The pod shall be operational at temperatures of maximum 95 degrees Celcius on outer skin and 152 degrees Celcius on leading egde for a maximum of 25 minutes.}\\
This test verifies that the pod is operational at the given temperatures and places for 25 min.\\

	\subparagraph{Preparations needed:}
	\begin{itemize}
	\item None.
	\end{itemize} 
		
	\subparagraph{Equipment required:}
	\begin{itemize}
	\item  A thermometer must be provided.
	\end{itemize}
	
\paragraph{ReqNo30-T}\mbox{}\\ % Ivan
Requirement: \textit{The system shall include a hardware implemented safety interlock to prevent dispensing on ground.}\\

This test verifies that the hardware implemented safety interlock prevents dispensing on ground.
	\subparagraph{Preparations needed:}
	\begin{itemize}
	\item The aircraft must be grounded.
	\item The mode must be set to manual mode.
	\end{itemize}

\paragraph{ReqNo31-T}\mbox{}\\ % Fatemeh
Requirement: \textit{The system shall provide a method of loading software to MWS.}\\
The system applied the method for loading software.
	\subparagraph{Preparations needed:}
	\begin{itemize}
	\item This has to be done before the other tests involving MWS.
	\item Different methods should be provided
	\end{itemize}

\paragraph{ReqNo35-T}\mbox{}\\ % Lasse
Requirement: \textit{The physical dimensions of the pod cannot exceed 0.5$\times$0.5$\times$5 meter.}
\\
The dimensions of the pod is measured using a measuring tape.

	\subparagraph{Preparations needed:}
	\begin{itemize}
	\item A measuring tape must be provided.
	\end{itemize}

\paragraph{ReqNo36-T}\mbox{}\\ % Lars N
Requirement: \textit{The aircraft has to be loaded with the payloads before take-off.}
\\
The aircraft is loaded with payloads before take-off. In air it is attempted to dispense the payloads. It is checked whether the loaded payloads are dispensed.
\\
	\subparagraph{Preparations needed:}
	\begin{itemize}
	\item None.
	\end{itemize} 

\paragraph{ReqNo37-T}\mbox{}\\ % Sergiu
Requirement: \textit{Pilots must be educated in handling the system from the cockpit.}
\\
This test verifies that the pilots are properly trained in handling the system from the cockpit.
\\
	\subparagraph{Preparations needed:}
	\begin{itemize}
	\item None.
	\end{itemize} 

\paragraph{ReqNo38-T}\mbox{}\\ % Lars J
Requirement: \textit{Technicians must be educated in maintenance of the system.}\\
This test verifies that the technicians are licensed in maintaining the system.
	\subparagraph{Preparations needed:}
	\begin{itemize}
	\item None.
	\end{itemize}

	\subparagraph{Equipment required:}
	\begin{itemize}
	\item None.
	\end{itemize}
	
\paragraph{ReqNo39-T}\mbox{}\\ % Ivan
Requirement: \textit{The chaffs and flares shall be transported in accordance to Military Standard Transportation and Movement Procedures (MILSTAMP).}\\

This test verifies that the chaffs and flares are transported in accordance to Military Standard Transportation and Movement Procedures (MILSTAMP).
	\subparagraph{Preparations needed:}
	\begin{itemize}
	\item None. 
	\end{itemize}

\paragraph{ReqNo40-T}\mbox{}\\ % Fatemeh
Requirement: \textit{The chaffs and flares shall be labeled and packed in accordance to MIL-STD-2073-1E}\\
Packing and labelling of chaffs and flares must comply with MIL-STD-2073-1E.

	\subparagraph{Preparations needed:}
	\begin{itemize}
	\item Packaging and labelling method should be provided. 
	\end{itemize}
